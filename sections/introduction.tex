\section{Introduction}

Topological data analysis is an emerging field of applied algebraic topology that provides noise-resistant methods of interpreting real-world data. The cornerstone of topological data analysis is \emph{persistent homology}, a technique introduced by Zomorodian and Carlsson \cite{zomorodian2005computing} which relates topological features of a data set over different levels of detail. Each feature can be assigned an interval that describes the levels of detail at which it appears. The main insight is that any important feature of the data set will appear over a large interval. 

The algebraic object underlying persistent homology is the \emph{persistence module}, which is a diagram of vector spaces indexed by a partially ordered set. In practice, this indexing set is a finite subset of the natural numbers. Section~\ref{section-standard} shows that persistence modules of this sort have a simple structure, and always decompose into a sum of interval modules. Each interval is interpreted as a topological feature. 

This set of intervals is known as the \emph{persistence diagram} associated with the persistence module. One of the most important results on persistence diagrams is the stability theorem, which states that small changes in the initial data lead to small changes in the persistence diagram. This fact justifies the persistence diagram's use as an invariant of real-world data.

In proving this theorem, we need to pass to persistence modules indexed over the real line. Here we face a problem: not every persistence module decomposes as a sum of interval modules. Sections~\ref{section-modules} and \ref{section-diagrams} of this thesis are dedicated to working around this issue and finding an appropriate definition for persistence diagrams in this setting. Here we follow arguments made by Chazal et al. \cite{chazal2012structure}. With this foundation in place, we then prove the stability theorem in Section~\ref{section-stability}.

Finally, Sections~\ref{section-category-theory} and \ref{section-observable} follow a recent paper by Chazal, Crawley-Boevey and de Silva \cite{chazal2014observable} to show that, in a very precise sense, the difficulties we face above are all localised to the behaviour of persistence modules over infinitesimally small intervals. We define the category of persistence modules and relate it to the much better behaved `observable' category. This allows us to prove a strong structure theorem for persistence modules over the real numbers.

The main contribution of this thesis is a self-contained treatment of persistence modules over the real line, collecting results from many different sources. A number of proofs have been expanded or simplified from those that appear in the literature.