\documentclass[12pt,a4paper]{article}
\usepackage[utf8]{inputenc}
\usepackage{amsmath}
\usepackage{amsfonts}
\usepackage{amssymb}
\author{Mitchell Riley}
\begin{document}
\begin{abstract}
Topological data analysis is an emerging field of applied algebraic topology that provides noise-resistant methods of interpreting real-world data. The cornerstone of topological data analysis is `persistent homology', a technique that relates topological features of a data set over different levels of detail. Each feature can be assigned an interval that describes the levels of detail at which it appears. The main insight is that any important feature of the data set will appear over a large interval.

The algebraic object underlying persistent homology is the `persistence module'. This talk will explain how persistence modules arise and give an overview of some of their basic properties. We will then define the main invariant used in persistent homology, the persistence diagram, and give examples of diagrams constructed from real-world data. Finally, we will discuss the celebrated stability theorem which guarantees that small changes in the data lead to small changes in the persistence diagram.
\end{abstract}

\end{document}